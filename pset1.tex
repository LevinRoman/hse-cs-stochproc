% !TEX encoding = UTF-8 Unicode
\documentclass[12pt]{article}
\usepackage[utf8]{inputenc}
\usepackage{color}
\usepackage[T2A]{fontenc}
\usepackage[utf8]{inputenc}
\usepackage[intlimits]{amsmath}
\usepackage{amssymb}
\usepackage{dsfont}
\usepackage{bm}
\usepackage{soul}
\usepackage[normalem]{ulem}
\usepackage[russian,english]{babel}
\usepackage{hyperref}
\hypersetup{
    colorlinks=true,
    linkcolor=blue,
    filecolor=magenta,      
    urlcolor=cyan,
}
\usepackage{mathtools}
\usepackage{graphicx}
\graphicspath{{Pictures/}}

\renewcommand{\labelenumi}{(\alph{enumi})} % Use letters for enumerate
% \DeclareMathOperator{\Sample}{Sample}
\let\vaccent=\v % rename builtin command \v{} to \vaccent{}
\usepackage{enumerate}
\renewcommand{\v}[1]{\ensuremath{\mathbf{#1}}} % for vectors
\newcommand{\gv}[1]{\ensuremath{\mbox{\boldmath$ #1 $}}} 
% for vectors of Greek letters
\newcommand{\uv}[1]{\ensuremath{\mathbf{\hat{#1}}}} % for unit vector
\newcommand{\abs}[1]{\left| #1 \right|} % for absolute value
\newcommand{\avg}[1]{\left< #1 \right>} % for average
\let\underdot=\d % rename builtin command \d{} to \underdot{}
\renewcommand{\d}[2]{\frac{d #1}{d #2}} % for derivatives
\newcommand{\dd}[2]{\frac{d^2 #1}{d #2^2}} % for double derivatives
\newcommand{\pd}[2]{\frac{\partial #1}{\partial #2}} 
% for partial derivatives
\newcommand{\pdd}[2]{\frac{\partial^2 #1}{\partial #2^2}} 
% for double partial derivatives
\newcommand{\pdc}[3]{\left( \frac{\partial #1}{\partial #2}
 \right)_{#3}} % for thermodynamic partial derivatives
\newcommand{\ket}[1]{\left| #1 \right>} % for Dirac bras
\newcommand{\bra}[1]{\left< #1 \right|} % for Dirac kets
\newcommand{\braket}[2]{\left< #1 \vphantom{#2} \right|
 \left. #2 \vphantom{#1} \right>} % for Dirac brackets
\newcommand{\matrixel}[3]{\left< #1 \vphantom{#2#3} \right|
 #2 \left| #3 \vphantom{#1#2} \right>} % for Dirac matrix elements
\newcommand{\grad}[1]{\gv{\nabla} #1} % for gradient
\let\divsymb=\div % rename builtin command \div to \divsymb
\renewcommand{\div}[1]{\gv{\nabla} \cdot \v{#1}} % for divergence
\newcommand{\curl}[1]{\gv{\nabla} \times \v{#1}} % for curl
\let\baraccent=\= % rename builtin command \= to \baraccent
\renewcommand{\=}[1]{\stackrel{#1}{=}} % for putting numbers above =
\providecommand{\wave}[1]{\v{\tilde{#1}}}
\providecommand{\fr}{\frac}
\providecommand{\RR}{\mathbb{R}}
\providecommand{\NN}{\mathbb{N}}
\providecommand{\seq}{\subseteq}
\providecommand{\e}{\varepsilon}


\begin{document}

{\noindent\Large\bf  \\[0.5\baselineskip] {\fontfamily{cmr}\selectfont  Домашнее задание I. Теория случайных процессов}         }\\[2\baselineskip] % Title
{ {\bf \fontfamily{cmr}\selectfont Вероятностные модели и статистика случайных процессов}\\
\vspace{0.3cm}
\\
{\textit{\fontfamily{cmr}\selectfont     \today}}}~~~~~~~~~~~~~~~~~~~~~~~~~~~~~~~~~~~~~~~~~~~~~~~~~~~~~~~~~~~~~~~~~~~~~~~~~~~~~    {\large \textsc{}} % Author name
\\[1.4\baselineskip]
\textbf{Дедлайн:} 2 февраля, 23:59\vspace{0.3cm}\\ 
\textbf{Указания:} Домашнее задание необходимо отправить до дедлайна на почту hse.cs.stochastics@gmail.com с темой "Домашнее задание 1. ФИО студента" в едином файле формата .pdf. Дедлайн жесткий и отправки после дедлайна оцениваться не будут.
\vspace{1cm} \\
Напомним, что вектор $X= (X_1, \dots, X_n)$ называется гауссовским, если для любого набора коэффициентов $(\lambda_1, \dots, \lambda_n) \in \mathbb{R}^n$ случайная величина $Y \stackrel{\text{def}}{=} \sum_{k=1}^n \lambda_k Y_k$ имеет нормальное распределение. 
\par Напомним определение Винеровского процесса: процесс $W_t$ называется Винеровским (или броуновским движением), если 
\begin{itemize}
    \item $W_0 = 0 \,\,\, \mathbb{P}$- п.н.
    \item $W_t$ имеет независимые приращения $\forall t$.
    \item $W_t - W_s \sim \mathcal{N}(0, t- s) \,\,\, \forall t > s \ge 0$.
\end{itemize}

\section{Гауссовские случайные процессы}
\textbf{Задача 1.1.} Какие из следующих ковариационных функций валидны (могут быть ковариационной функцией случайного процесса)?. Ответ обосновать.
\begin{align}
    R_1(t, s) = \min\{t, s\} - ts \qquad R_2(t, s) = \min\{t, s\} - t(s + 1)
\end{align}
\textbf{Задача 1.2.} Докажите эквивалентность следующих утверждений: \\
(i) Характеристическая функция вектора $X$ выглядит таким образом 
\begin{align}
    \varphi_X(\mathbf{u}) = \exp\left\{i \mathbf{u}^T \cdot \mu - \frac 1 2 \mathbf{u}^T \Sigma \mathbf{u} \right\},
\end{align}
где $\mu \in \mathbb{R}^n$ и $\Sigma$ симметричная неотрицательно определённая квадратная матрица размера $n \times n$. \\
(ii) Вектор $X$ представим в виде 
\begin{align}
    X = A X^{\circ} + \mu,
\end{align}
где $\mu \in \mathbb{R}^n$, $A \in \mathbb{M}_n(\mathbb{R})$, а $X^{\circ} \in \mathbb{R}^n$ --- стандартный нормальный вектор, т.е. все компоненты этого вектора независимы в совокупности и распределены по закону $\mathcal{N}(0, 1)$.   
\\
\textbf{Задача 1.3.} Рассмотрим следующую вариацию определения Винеровского процесса. 
\par Винеровский процесс (броуновское движение) - это гауссовский процесс $W_t$ с математическим ожиданием $m(t) = 0$ и кова- риационной функцией $R(s, t) = min\{s, t\}$. 
\par Докажите эквивалентность этих двух определений. 
\\
\textbf{Задача 1.4.} Найдите 
\begin{enumerate}
    \item 
    \begin{align}
        \lim_{n \to \infty} \sum_{i=1}^n \left(W_{t_i} - W_{t_{i-1}}\right)^2 
    \end{align}
    \item
    \begin{align}
         \lim_{n \to \infty} \sum_{i=1}^n \left|W_{t_i} - W_{t_{i-1}}\right|
    \end{align}
    для Винеровского процесса на отрезке $[0, t]$:
\end{enumerate}
\textbf{Задача 1.5.} Пусть $N_t$ - неоднородный процесс Пуассона с интенсивностью $\lambda(t)$. Докажите, что 
\begin{itemize}
    \item функция $\Lambda(t) = \int_{0}^t \lambda(s) \, ds$ имеет обратную. 
    \item процесс $M_t = N_{\Lambda^{-1}(t)}$ является однородным Пуассоновским процессом. 
\end{itemize}
\section{Эксперимент}

\textbf{Задача 1.6.} Написать программу, которая генерирует реализации фрактального броуновского движения, с помощью вычисления стохастического интеграла по броуновскому движению
\\
\end{document}